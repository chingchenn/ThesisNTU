% !TeX root = ../main.tex

\chapter{Numerical modelling method}


計算機模擬是在計算機上執行數學建模的過程,被用來預測物理系統與現實世界的行為與結果。透過計算機模擬可以在成本較少的狀態下實現結果的推斷,量化模型實現後的不確定性,因此,計算機模擬已經大量被運用在社會科學、醫療科學、工程學與自然科學上。自板塊構造學說被提出以來,實現動態地球的隱沒帶無論在巨觀尺度下構造的演化,界觀尺度下岩石的變形狀態或微觀尺度下礦物的排列與物質置換皆是人類致力於研究的目標。由於隱沒過程牽扯到緩慢的岩石力學變形,並且目前人類對於深部岩石圈的演化過程以及深部的地質作用(例如相變、脫水與部分熔融)尚未完全了解,目前有多種方法被用以探討隱沒帶中內部機制。其中,地質建模與模擬是探討地球內部演化常見的方法之一,常見的有沙箱模型與數值模型等。在地球動力學建模中,我們將岩石單元視為地質構造上的連續介質,且岩石具有巨觀物理量,受牛頓力學所支配。由於在地質尺度上長期且緩慢的岩石變形過程可視為流體,因此地球動力學過程通常被視為流體力學的展現。描述流體力學現象的數學式共有三大守恆定律,分別為質量守恆、動量守恆與能量守恆。描述連續介質特性的方程式

該三大守恆定律可寫成偏微分方程式的形式,此時因偏微分方程式之解析解多半不存在,僅能利用數值近似方法求得最佳解。地球動力學上的數值模擬為利用數值方法與流體力學方程式求解的模擬方法,是研究地球內部演化過程的重要工具。

本研究利用數值模擬的方法以解決平坦隱沒的動力學過程,其中,海洋岩石圈隱沒至大陸岩石圈之下。本研究的目的為探討平坦隱沒形成的條件以及演化發育過程。

在本章節中會先介紹數值模擬的計算方式以及模型中的假設,接著說明初始模型的設定。

\section{Governing equation}

\subsection{Continuum---Conservation of mass}

在地球動力學建模中,我們將岩石單元視為地質構造上是連續介質,且岩石具有巨觀物理量。練續性的描述被用於岩石的密度、壓力、速度、應變等不同維度物理的場變量(field variables)。連續性可視為定量的數學形式,即連續方程式。

其中,拉格朗日的質量守恆方程式由下式所表示:

\begin{align}
\frac{\partial \sigma_{ij}}{\partial x_j}+\rho g_i = \rho \frac{\partial D_{vi}}{\partial t} 
\end{align}

在流體力學中,描述流場性質的方法主要有兩種,第一為尤拉描述法,第二為拉格朗日描述法。在本研究中所使用的描述方法為拉格朗日法。每個拉格朗日點被嚴格的連到一個單一的物質點上,並且會隨著該點移動。因此,同一個質點總是在同一格座標上,與時間無關。反觀若為 Eulerian point, 則是一個固定的不動點。

\begin{figure*}[ht!]
    \centering
    \includegraphics[width=4.5in]{eulerian_and_lagrangian.pdf}
    \caption{ Eulerian (a) and Lagrangian (b) elementary volumes considered for the derivation of continuity equation. }
    \label{fig::Lagrangian Volume}
\end{figure*}

公式(2.1)的證明如下:

由於拉格朗日描述法中同樣的物質點永遠在相同的位置上,因此拉格朗日網格的質量永遠相同,但網格體積會因網格內在膨脹與收所作用而隨時間改變。另一拉格往日網格體積為$V_0$,其初始平均密度$\rho_0$為:

\begin{align}
\rho_0 = \frac{m}{\Delta x_0 \Delta y_0 \Delta z_0}
\end{align}

經過一段短周期$\Delta t$時間後,受力之拉格朗日體積承受網格內內在作用,因此其體積變成$V_1$,而其平均密度$\rho_1$變成:
\begin{align}
\rho_1 = \frac{m}{\Delta x_1 \Delta y_1 \Delta z_1}
\end{align}
對拉格朗日流場所有自變數隨時間變化可表示為:
\begin{align}
\frac{D\rho}{Dt} \approx \frac{\Delta \rho}{\Delta t} = \frac{\rho_1-\rho_0}{\Delta t}=\frac{m}{\Delta x_1 \Delta y_1 \Delta z_1\Delta t}-\frac{m}{\Delta x_0 \Delta y_0 \Delta z_0\Delta t}\label{eqn:Drho-Dt}
\end{align}
在新舊拉格朗日體積中的關係可視為體積邊界的相對運動,由於位移等同於速度乘以持續時間:
\begin{align}
\Delta x_1 = \Delta x_0+\Delta t\Delta v_x\label{eqn:disx}\\ 
\Delta y_1 = \Delta y_0+\Delta t\Delta v_y\label{eqn:disy}\\
\Delta z_1 = \Delta z_0+\Delta t\Delta v_z\label{eqn:disz}
\end{align}
將上式 ($\ref{eqn:disx}$)-($\ref{eqn:disz}$)帶入(\ref{eqn:Drho-Dt})中可得:

\begin{align}
\frac{D\rho}{Dt} \approx \frac{\Delta \rho}{\Delta t} =\frac{m\Delta x_0 \Delta y_0 \Delta z_0-m\Delta x_1 \Delta y_1 \Delta z_1}{\Delta x_1 \Delta y_1 \Delta z_1\Delta t\Delta x_0 \Delta y_0 \Delta z_0}\label{eqn:Drho-Dt2}
\end{align}
由於$\Delta x_0 \Delta y_0 \Delta z_0=\rho_0$, 可獲得下列數學式:
\begin{align}
\frac{\Delta \rho}{\Delta t}+\rho_0\frac{\frac{\Delta v_x}{\Delta x_0}+\frac{\Delta v_y}{\Delta y_0}+\frac{\Delta v_z}{\Delta z_0}+K_1}{K_2} = 0
\end{align}
\begin{align}
K_1 = \Delta t(\frac{\Delta v_x}{\Delta x_0}\frac{\Delta v_y}{\Delta y_0}+\frac{\Delta v_x}{\Delta x_0}\frac{\Delta v_z}{\Delta z_0}+\frac{\Delta v_y}{\Delta y_0}\frac{\Delta v_z}{\Delta z_0}+\Delta t \frac{\Delta v_x}{\Delta x_0}\frac{\Delta v_y}{\Delta y_0}\frac{\Delta v_z}{\Delta z_0})
\end{align}
\begin{align}
K_2=(1+\Delta t\frac{\Delta v_x}{\Delta x_0})(1+\Delta t\frac{\Delta v_y}{\Delta y_0})(1+\Delta t\frac{\Delta v_z}{\Delta z_0})
\end{align}
其中$K_1$ 與 $K_2$ 係數在$\Delta t$趨近於$0$時分別為$0$與單位,因此可得:
\begin{align}
\frac{D\rho}{Dt}+\rho\frac{\partial v_x}{\partial x}+\rho\frac{\partial v_y}{\partial y}+\rho\frac{\partial v_z}{\partial z} = 0
\end{align}
或
\begin{align}
\frac{D\rho}{Dt}+\rho div(\vec v) = 0
\end{align}
在地球動力學模擬中,因為視為流體的岩石流動速度遠低於音速,且在上部地函以上區域的體壇性系數遠大於壓力,因此可將岩石視為不可壓縮物質。此時,
While in geodynamics modelling, the density variations are small enough to be ignored, which is the result of Boussinesq approximation. The Boussinesq approximation assume that the density is linear proportional to the temperature and the small density variation is then neglected, except the gravity term.
\begin{align}
\rho (T) = \rho_0[1-\alpha (T-T_0)] 
\end{align}
where $\rho_0$ is the reference density at temperature $T_0$ and $\alpha$ is the volumetric thermal expansion coefficient. The boussinesq approximation also represent the incompressible condition, which mean the density of material points does no change with time. There are two condition that can satisfy the incompressible : First, the velocity of materials must much smaller than the acoustic velocity. Second, the bulk modulus of material must much larger than pressure. The incompressible continuity equation is broadly used in numerical geodynamic modelling. 
\begin{align}
\nabla \cdot (\vec v) = 0 
\end{align}
Eq. (2.15) is the conservation of mass in our numerical modelling approach.

\subsection{Motion---Conservation of momentum}
In geodynamics, the time-dependent phenomena involve deformation of continuous media, which is the effect of the balance of internal and external forces that act in these media. So as to relate forces and deformation, an equation of motion may be used --The momentum equation. The momentum equation is a differential equivalent of Newton’s second law to a continuous medium.
\begin{align}
f=ma
\end{align}
$f$ is the net force acting on the object and $m$ is the mass of material. The momentum equation for a continuous medium in the gravity field:
Eulerian Form:  \begin{align}
\frac{\partial \sigma_{ij}}{\partial x_j}+\rho g_i = \rho (\frac{\partial v_i}{\partial t}+v_j\frac{\partial v_i}{\partial x_j})
\end{align}
Lagrangian Form:  \begin{align}
\frac{\partial \sigma_{ij}}{\partial x_j}+\rho g_i = \rho \frac{\partial D_{vi}}{\partial t}
\end{align}
\begin{figure*}[ht!]
    \centering
    \includegraphics[width=4in]{momentum.pdf}
    \caption{ Lagrangian elementary Volume considered for the derivation of the respective form of x-momentum equation.}
    \label{fig::Lagrangian Volume}
\end{figure*}

While considering a small Lagrangian volume, the net force acting on the object which can be computed locally. We will proof the momentum equation of Lagrangian Form below:

For x-component
\begin{align}
f_x=f_{xA}+f_{xB}+f_{xC}+f_{xD}+f_{xE}+f_{xF}+mg_x 
\end{align}
$f_{xA}- f_{xF}$ are stress-related forces, from the outside of the volume on the respective boundaries A-F. 
$f_g=m_gx$ is the gravity force.
\begin{align}
f_{xA} = -\sigma_{xxA}\Delta y\Delta z\\
f_{xB} = +\sigma_{xxB}\Delta y\Delta z\\
f_{xC} = -\sigma_{xyC}\Delta x\Delta z\\
f_{xD} = +\sigma_{xyD}\Delta x\Delta z\\
f_{xE} = -\sigma_{xzE}\Delta x\Delta y\\
f_{xF} = +\sigma_{xzF}\Delta x\Delta y
\end{align}
We replace the force Eq(2.20-2.25) to Eq (2.19):
\begin{align}
(\sigma_{xxB}-\sigma_{xxA})\Delta y\Delta z+(\sigma_{xyD}-\sigma_{xyC})\Delta x\Delta z+(\sigma_{xzF}-\sigma_{xzE})\Delta x\Delta y+mg_x = ma_x 
\end{align}
Normalising both sides by considered Lagrangian volume
\begin{align}
V=\Delta x\Delta y\Delta z
\end{align}
we obtain
\begin{align}
\frac{(\sigma_{xxB}-\sigma_{xxA})\Delta y\Delta z}{V}+\frac{(\sigma_{xyD}-\sigma_{xyC})\Delta x\Delta z}{V}+\frac{(\sigma_{xzF}-\sigma_{xzE})\Delta x\Delta y}{V}+\frac{m}{V}g_x=\frac{m}{V}a_x
\end{align}
or
\begin{align}
\frac{\Delta\sigma_{xx}}{\Delta x}+\frac{\Delta\sigma_{xy}}{\Delta y}+\frac{\Delta\sigma_{xz}}{\Delta z}+\rho g_x = \rho a_x
\end{align}
While the differences of the respective stresses components all tend to zero, we obtain:
\begin{align}
\frac{\partial\sigma_{xx}}{\partial x}+\frac{\partial\sigma_{xy}}{\partial y}+\frac{\partial\sigma_{xz}}{\partial z}+\rho g_x = \rho a_x
\end{align}
or 
\begin{align}
\partial_j\sigma_{ij}+\rho g_i = \rho \ddot u
\end{align}
where $u$ is the displacement.
\subsection{Heat equation --- Conservation of energy}

To describe the balance of energy in a continuum material, heat equation is apply to measure the temperature change. The heat equation sloved the heat transport and porvided the temperature field. Below is the heat equation in Lagrangian form :
\begin{align}
\rho C_p \frac{DT}{Dt} = -\frac{\partial q_x}{\partial x}-\frac{\partial q_y}{\partial y}-\frac{\partial q_z}{\partial z}+H_s+H_L
\end{align}

where $\rho$ is the density, $C_p$ is the heat capacity at constant pressure (isobaric heat capacity), $H_s$ is shear heating and $H_L$ is the latent heat production.

The proof of heat equation is show below:

Base on the Boussinesq approximation, the imcompressible Lagrangian form governing equations are:
\begin{align}
\nabla \cdot (\vec v) = 0 
\frac{\partial \sigma_{ij}}{\partial x_j}+\rho g_i = \rho \frac{\partial D_{vi}}{\partial t}
\rho C_p \frac{DT}{Dt} = -\frac{\partial q_x}{\partial x}-\frac{\partial q_y}{\partial y}-\frac{\partial q_z}{\partial z}+H_s+H_L
\end{align}

Describe conservation of mass, conservation of momentum and conservation of energy, respectively.


\section{Finite elements method}

finite elements method

\section{Rheological behavior}

**What is viscous and the viscous rheology of rock**

In the near-surface region, rocks undergo relatively low temperature and, therefore, the Earth's lithosphere easily result in brittle (at low pressure) and plastic (at high pressure) deformation. 
While in the deep earth, temperature increasing with depth, rocks behave viscous with irreversible deformation. 
Therefore, if a geodynamics model need to account for a wide range of rocks properties, it should consider the elasto-visco-plastic rheology of rocks.

Elastic rheology assume that the relationship between applied stress and strain is proportionality. 

\begin{figure*}[ht!]
    \centering
    \includegraphics[width=3in]{elasticdeformaiton.pdf}
    \caption{ Elastic deformation  }
    \label{fig::elastic}
\end{figure*}

While plastic.......


Viscous rheology define....

\section{Phase change}

In this study, we use markers to trace the rocks phases, pressure and temperature. 
At the time the pressure and temperature satisfy the phase transformation condition, the marker will turns to new rock phase. 
A basalt element represent an element with more than 30 presents of the basalt phase markers. 
In this case, the element behave the deformation as same as the basalt rheology.

\subsection{Peridotite --- Serpentinite}

Once the subducting plate sink into mantle, sediment on oceanic plate undergo higher pressure and temperature that release a large amount of fluids. 
On the other hand, the oceanic plate itself also carries seawater into the mantle. 
Fluids in subduction zone mostly concentrated in the mantle wedge. 
The dry mantle wedge undergo hydration process, lead to the transformation of peridotite to serpentinite.  
The serpentinite depth and thickness in subduction zone is not well understand since the seismic study constrain still contain high uncertainly, we model the phase transformation process of serpentinite in parameter way.     

Serpentinite are stable in the colder mantle wedge relative to deeper mantle, and therefore once the serpentinite under unstable field, we assume that serpentinite rocks release fluid and transfer to peridotite. 
The following equations are the conditional expressions of serpentinite---peridotite transformation. Figure is the phase diagram of mantle phases.


\subsection{Basalt --- Eclogite}

As the oceanic crust sink into deeper mantle, the mafic rocks enters the eclogite stability field in the condition of high pressure. 
Therefore, basalt phases transform to eclogite.  
In this model, oceanic crust are tracked and compared with the eclogite stability filed, the following equations are the conditional expressions of mafic rocks transformation. 
Figure below is the mafic rocks phase diagram.
\begin{figure*}[ht!]
    \centering
    \includegraphics[width=3in]{basalt_phase_diagram.pdf}
    \caption{ Phase diagram showing the stability field for mafic rocks (Hacker et al., 2003).  }
    \label{fig::elastic}
\end{figure*}

\subsection{Sediment --- Schist}

Once sediment undergo higher pressure, the compression process and mataphase(變質作用)process occur.
In this model, the following equations are the conditional expressions of transformation process of sediment to schist.

$T > 650^{\circ} C$\\
$depth >  20 $km 

The subducted sediments will turn into schist when temperature is greater than 650$^\circ$C and pressure is greater than (a number)

\subsection{Hydrated olivine --- Peridotite}

We considering a hydrated peridotite under the oceanic crust in our model. 
The magma will only generated above the hydrated subducting oceanic lithosphere.
Once the tempertaure is too high to make the rock contain water, the hydrated olivine transform to normal perodotite.
The following equation is the conditional expression of this transformation.

$T > 800-35\times 10^{-9}\times (depth-62)^{2\circ}C$

\section{Boundary condition}

\subsection{Kinematic boundary condition}

\subsection{Thermal boundary condition}

For oceanic lithosphere, we used half space cooling model in our model to defined the thermal condition. 
The plate depth is proportional to the square root of oceanic lithosphere age, and therefore, the half space cooling model can predict well for the temperature of the oceanic plate.
Follow by David and Lister, 1974(mention in Stein, 1995):

$T=T_m\cdot {erf}(\frac{z}{2\sqrt{\kappa t}})$

$T$ is the temperature, $T_m$ is the mantle temperature, in this model the temperature is 1330 K,
$z$ is the depth from surface in kilometer and $\kappa$ is the thermal diffusivity coefficient, that is, $10^{-6}$ in this study.
$t$ is the lithosphere age in Myr.


For continental lithosphere, the thermal condition is defined in linearly.


\section{FLAC}

We used the Fast Lagrangian Analysis of Continua (FLAC) technique. FLAC is a two-dimensional, explicit finite element with Lagrangian gird of numerical program.

What is explicit?

An explicit form is any solution that is given in the form $y=y(t)$. That is, $y$ only shows up once on the ledt hand side and is only in first power. Ob the other hand, implicit form is any solution that is not in explicit form. In finite element method, explicit solution solve the acceleration. In most cases, only the diagonal elements in matrix are not equal to zero, in other words, it is simple to solve an inversion solution form inverse matrix. 

Gird?



\section{Initial Model}

initial model


