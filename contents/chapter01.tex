% !TeX root = ../main.tex

\chapter{Introduction}

\section{Background}

Geosynclinal theory was one of the famous theory in the geological studies history. It was believed that the earth's crust deformed with vertical motion in each geoloical zone while the Plate tectonic theory claimed that the earth's crust deformed with horizontal motion. In Geosynclinal theory, geological zone occurred weathering, magmatism and matamporphism process during geological time scale. Since 1960s, geophysics was revolutionized by the discovery of plate tectonics, that is, the plate tectonic theory had been proposed. The heat heterogeneous in earth's interior lead to gravitational instability, and therefore the earth's surface manifestations as plate tectonics (Jordan, 1978). The plate tectonic theory first defined the uppermost layers which is outside the upper thermal boundary layer --- lithosphere which consist with crust and part of mantle. Since lithosphere broken into many plate, horizontal motion dominate the Earth's surface with tectonic deformation processes  occur in plate boundary. The convergent boundaries represented by the trench and subduction zone where one of the plate destroyed. The divergent boundaries represented by the mid-ocean-ridge system where plate produced. The transform faults os a famous example of which plates move laterally relative to each other (The solid earth, 2005). 

In the convergent plate boundary, the thermal state of old lithosphere is relative colder than the ambient mantle, which lead to a denser condition. The cold region is enough to generate a gravitational instability in the subduction zone. Therefore, the heavy lithosphere develop the trench and sink into mantle which is so-called "slab".  The main changes which occur in the subducting plate are the shallow reaction of the oceanic crust to eclogite and the changes deeper in the mantle of olivine to a spinel structure and then to post-spinel structures. Phase changes result in increases in the density of the subducting slab. In the meantime, slab remains cold with gravitational instability. Thermal contraction provides the greatest contribution to the overall driving force. (The solid earth, 2005; Turcotte and Schubert, 2002). The geometry of slab varies considerably on Earth, with variations in slab dip angle and bending curvature (Schellart, 2020).

Flat subduction is a tricker circumstances of subduction geometry. In the flat subduction area, a part of the slab attains a horizontal orientation for several hundred kilometers below the overriding plate while the others part of the slab sinks into the mantle with a normal slab dip.



\section{Review of flat subduction numerical models}



\section{Motivation}

\section{Geophysical observation in Cocos subduction zone}

\section{Summery}
