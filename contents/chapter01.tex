% !TeX root = ../main.tex

\chapter{Introduction}

\section{Background}

Geosynclinal theory was one of the famous theory in the geological studies history. 
It was believed that the earth's crust deformed with vertical motion in each geoloical zone while the Plate tectonic theory claimed that the earth's crust deformed with horizontal motion. 
In Geosynclinal theory, geological zone occurred weathering, magmatism and matamporphism process during geological timescale. 
Since 1960s, geophysics was revolutionized by a series of discoveries from continental drift, seafloor spreading to plate tectonics, that is, the theory of plate tectonic had been proposed. 
The heat heterogeneous in earth's interior lead to gravitational instability, and therefore the earth's surface manifestations as plate tectonics (Jordan, 1978). 
The plate tectonic theory first defined the uppermost layers which is outside the upper thermal boundary layer --- lithosphere which consist with crust and part of mantle. 
Since lithosphere broken into many plate, horizontal motion dominates the Earth's surface with tectonic deformation processes  occur in plate boundary. 
The convergent boundaries represented by the trench and subduction zone where one of the plate destroyed. 
The divergent boundaries represented by the mid-ocean-ridge system where plate produced. 
The transform faults os a famous example of which plates move laterally relative to each other (The solid earth, 2005). 



In the convergent plate boundary, the thermal state of old lithosphere is relative colder than the ambient mantle, which lead to a denser condition. The cold region is enough to generate a gravitational instability in the subduction zone. Therefore, the heavy lithosphere develop the trench and sink into mantle which is so-called "slab".  The main changes which occur in the subducting plate are the shallow reaction of the oceanic crust to eclogite and the changes deeper in the mantle of olivine to a spinel structure and then to post-spinel structures. Phase changes result in increases in the density of the subducting slab. In the meantime, slab remains cold with gravitational instability. Thermal contraction provides the greatest contribution to the overall driving force. (The solid earth, 2005; Turcotte and Schubert, 2002). The geometry of slab varies considerably on Earth, with variations in slab dip angle and bending curvature (Schellart, 2020).
==
Flat subduction is a tricker circumstances of subduction geometry. In the flat subduction area, a part of the slab attains a horizontal orientation for several hundred kilometers below the overriding plate while the others part of the slab sinks into the mantle with a normal slab dip.



\section{Review of flat subduction numerical models}

目前造成平坦隱沒發生的機制眾說紛紜。南美洲區域平坦隱沒的發生區域與隱沒的中洋脊有幾何上的相關性,海洋地殼上中洋脊與海洋高原的存在可能會導致總體密度較低、浮力較大,因此過去曾經隱沒的中洋脊被認為是造成平坦隱沒的主要原因。

Hunen et al., 2002最早將模型加入增厚的海洋地殼,以模擬過去智利與秘魯曾經有中洋脊與海洋高原進入隱沒帶中的紀錄。增厚海洋地殼有較低的密度與較大的浮力,其上玄武岩相需要比原先更大的壓力與更高的溫度才會成為密度高的榴輝岩,可能使隱沒板塊與周遭地幔沒有顯著密度差而發生平坦隱沒。不過由於該研究模型僅二維,單純加入增厚海洋地殼所呈現的模型雖然能呈現平坦隱沒,但結果是假設第三維上有無限延伸的增厚海洋地殼,現實中增厚的海洋地殼能造成的浮力效應應遠小於二維模型中的結果。Florez-Rodr´ıguez et al., 2019在三維模型中證明了這一點,他們提出若將現在自然界中最大的洋脊隱沒進入地幔,其所提供的浮力也只會造成海洋板塊傾角減少原先的10度。若從自然界中來看,確實有許多區域皆有海脊隱沒的證據,例如勘察加半島(Kamchatka)有皇帝海脊(Emperor Ridge)隱沒、琉球(Ryukyu)有大東海脊(Daito Ridge)隱沒以及馬里亞納(Mariana)與馬庫斯—內克海脊(Marcus-Necker Ridge)隱沒,然而只有秘魯與智利有平坦隱沒的特徵。此外,在墨西哥有平坦隱沒的特徵,然而墨西哥沒有任何海脊或海洋高原的隱沒紀錄,因此增厚的海洋地殼發生平坦隱沒的理論近年來逐漸站不住腳(Schellart, 2020)。

Hunen et al., 2000 使用二維笛卡爾座標數值模型進行祕魯與智利平坦隱沒的模擬。在他們得模型中,唯一能成功演化出平坦隱沒的機制只有海溝後撤迫使大陸岩石圈逆衝到隱沒板塊之上。Liu and Currie, 2016 使用二維模型模擬過去古法拉隆板塊板塊的平坦隱沒機制,他們加入增厚的海洋地殼後並無法觸發平坦隱沒的產生,然而,再加入額外大陸岩石圈的水平速度後,平坦隱沒便能成功再現。Axen et al., 2018使用同樣的數值模型將古代北美西部的克拉通放置於大陸板塊測,成功模擬出增厚海洋地殼加上快速移動大陸岩石圈能發生平坦隱沒,並且能將克拉通從大陸岩石圈底部刮除,證實了平坦隱沒能破壞大陸岩石圈。在該研究中並沒有考慮克拉通對平坦隱沒的影響。

Manea et al., 2012提出了另外的看法。他們利用三維模型模擬過去30Ma以來智利區域的隱沒帶動態行為,使用額外施加的邊界條件強迫智利海溝後撤,發現海溝後撤能夠施加給隱沒板塊的地幔流吸力(suction)不足以讓巨大厚重的海洋板塊變平坦,因此他們在模型上覆板塊加上克拉通,系統性測試從150-300公里厚的大陸岩石圈與海溝距離600-1000公里時隱沒帶下方地幔流產生的動力壓力(dynamic pressure)。他們發現在只有在克拉通與海溝距離約800公里且克拉通厚度大於200公里時平坦隱沒才會生成。當他們把造成海溝後撤的邊界力移除時,不會觸發平坦隱沒的形成,因此他們得出的結論是需要同時有海溝後撤與克拉通的存在才會觸發平坦隱沒。這是首次將克拉通加進數值模型裡的平坦隱沒模型。隨後Liu and Currie, 2016效仿同樣的機制,將過去普遍認為存在於北美板塊西部下方的科羅拉多高原山根放入模型中,模擬古法拉龍板塊平坦隱沒演化。他們認為克拉通與山根的存在只是加快平坦隱沒的形成,但真正觸發平坦隱沒的機制是增厚海洋地殼延緩玄武岩相變成榴輝岩。Hu et al., 2016使用三維模型CitcomS模擬整個南美洲海溝45 Ma以來隱沒帶演化。在加入克拉通的模型中,隱沒板塊傾角有降低的趨勢,不過根據模型結果,真正造成平坦隱沒的形成依然與隱沒海脊相關,只有在海脊進入三維模型後隱沒傾角才出現顯著降低。

因此,目前的平坦隱沒數值模型大多以擬合智利、祕魯與法拉龍板塊為主,觸發平坦隱沒的機制大多與克拉通的存在與否、是否有洋脊隱沒以及上覆板塊的移動速度為主要測試,墨西哥區域尚未有平坦隱沒的數值模型被提出。在墨西哥,隱沒板塊上沒有任何增厚的紀錄,此外該地區北美板塊移動速率遠低於南美洲與過去法拉龍板塊隱沒時期的北美板塊,因此墨西哥區域的平坦隱沒機制尚未有統一定論。本研究重新模擬南美洲平坦隱沒的形成,探討平坦隱沒的形成機制,此外本研究期待能利用數值模擬得到墨西哥平坦隱沒的演化,填補過去尚未成熟的平坦隱沒機制理論。


\section{Motivation}

\section{Geophysical observation in Cocos subduction zone}

\section{Summery}
