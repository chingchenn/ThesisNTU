% !TeX root = ../main.tex

\chapter{結論}

本研究利用數值模型模擬智利與墨西哥的平坦隱沒地區熱構造結果,並討論可能的平坦隱沒形成機制,獲得以下結論:

1. 智利參考模型的平坦隱沒的深度、岩漿作用以及於模型地表展現壓縮特徵,皆與觀測資料有很好的吻合。
本研究支持智利平坦隱沒區域的形成原因可以在不具有隱沒洋脊的情況下實現。
本研究利用改變模型中蛇紋岩量模擬隱沒帶脫水作用程度,以及利用改變模型中岩將產生速率模擬不活躍的岩漿作用,在隱沒帶中形成較高的黏滯度的地函楔,導致地函動水壓力(吸力)力矩足夠與重力力矩抗衡,形成平坦隱沒。

2. 墨西哥參考模型平坦隱沒呈現一平坦段直接位於莫荷面底下,中間夾雜沉積物與少量地函蛇紋岩的弱物質,為過去觀測資料的假想提供新的約束。
墨西哥參考模型中需要具備地溫梯度較高的上覆板塊才能促使平坦隱沒的發育,本研究認為較高的上覆板塊溫度導致岩石圈強度下降,隱沒板塊容易將弱的地函岩石圈推開,又加上隱沒地殼年紀輕,因此平坦隱沒得以順利形成。

3. 本研究兩種平坦隱沒參考模型中,平坦隱沒段上皆存在玄武岩至榴輝岩的相變過程。
儘管榴輝岩密度大於地函,然而參考模型中,動水壓力力矩可大過具有榴輝岩的隱沒板塊重力力矩,足以形成平坦隱沒。


4. 墨西哥參考模型在地表距海溝250公里處出現隱沒板塊熔融所產生的埃達克岩火山島弧,確切證明隱沒板塊的熔融概念模型的可行性。
然而智利參考模型顯示在南美洲上方的埃達克岩不是隱沒板塊熔融的產物。

5. 在平坦隱沒生成環境下,越高溫的上覆板塊會形成平坦段長度越長且平坦段深度越淺的平坦隱沒,亦即平坦隱沒的長度與深度可能很大程度上與上覆板塊的溫度狀態有關。

6. 脫水作用與岩漿作用會降低地函楔中的黏滯度,導致軟流圈容易流入岩石圈地函楔低壓區,地函流吸力減少,進而降低動水壓力力矩。
過去對地函吸力的估計並沒有考量脫水作用與岩漿作用,因此,過去研究可能高估隱沒板塊的吸力。