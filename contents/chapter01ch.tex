% !TeX root = ../main.tex

\chapter{緒論}

地球的動力學相當複雜。地球動力學由地球與動力兩個詞所組成,顧名思義為探討地球內部的動力過程,以及地球內部物質受力後所發生的變形作用。地質學之父查理斯萊爾爵士(Sir Charles Lyell)在<<地質學原理(Principles of Geology)>>一書中提出的一句話「現在是通往過去的一把鑰匙 The present is the key to the past 」,表示過去所發生的地質事件與現在進行中的地質作用皆相同。

人類在很早之前便知道地球有動態過程,例如地震與火山爆發。早期的地球科學研究侷限於地表觀察,儘管在牛頓力學成為自然科學的顯學後,以物理基礎定量描述自然現象與作用力已被廣泛應用,但由於地質作用之時間尺度遠超乎當時可訂年之技術,因此地球內部之構造活動尚無人知曉。 直到19世紀開始,隨著人類在實驗與測量技術上的突破,藉由物理方法探測地球內部的技術才開始被應用到對地球內部的觀察,不過當時的地球物理學門與地質科學是完全分開的領域。直到板塊構造學說被提出後,地球科學從定性的地質描述轉變成定量的物理展現,所有地質現象在時間與空間上有最大程度上的整合。自從第一篇2D 隱沒帶數值模型文章(Minear and Toksoz, 1970)發布以來,地球動力學研究已經是發展成熟且被廣泛應用的地球科學學門。
\section{Background}

地球科學的科學革命始於1960年代提出的板塊構造學說,將地球最外層之剛性殼體---岩石圈視為地表水平運動單位。其因地球內部的熱引起重力不穩定而在軟流圈上水平運動(Jordan, 1978)。地球的岩石圈斷裂成許多剛性塊體,該塊體被稱為板塊。大部分由板塊水平運動所引起的變形作用發生在板塊邊界。在聚合板塊邊界,板塊發生破壞,包含碰撞與隱沒; 在分離板塊邊界,板塊發生增生,代表構造為海底擴張;錯動邊界中板塊不會顯著發生增生與破壞,代表構造為轉型斷層(Fowler, 2005)。全球主要的板塊邊界見圖1-1。

In the convergent plate boundary, the thermal state of old lithosphere is relative colder than the ambient mantle, which lead to a denser condition. The cold region is enough to generate a gravitational instability in the subduction zone. Therefore, the heavy lithosphere develop the trench and sink into mantle which is so-called "slab". As the subduction zone brings cold material to the deep mantle, 隨著隱沒板塊帶著較冷物質進入地幔深處,周圍壓力逐漸上升,岩石發生相變,隱沒板塊因成分與溫度與地幔物質不同,溫度上的差異造成更大的重力不穩定。在同等深度下,隱沒板塊的密度始終比周圍地幔高,為了隱沒系統的平衡,隱沒板塊持續下沉進入地球內部更深處,整段不穩定區域稱為隱沒帶不穩定。同時,隱沒板塊上的聚合板塊稱為上覆板塊。隱沒板塊為板塊移動與張裂的主要驅動力(Turcotte and Schubert, 2002),在隱沒過程中,岩石圈物質與軟流圈物質發生交互作用,隱沒板塊將海水與沈積物進入地幔中,降低地幔熔點,在上覆板塊側發生岩漿作用。自然界中上部地幔以上之隱沒幾何剖面有相當大的相異性,有許多原因影響隱沒傾角與隱沒曲率(Schellart, 2020)。
平坦隱沒(Flat slab subduction)是一種特殊的隱沒帶。
 In the flat subduction area, a part of the slab attains a horizontal orientation for several hundred kilometers below the overriding plate while the others part of the slab sinks into the mantle with a normal slab dip.

大部分活躍的隱沒帶中,若僅考慮地表至200公里深的幾何構造,隱沒板塊的轉樞點接近海溝且下凹,並且只有一個轉樞點,如圖1-2 (A)(B)。在部分區域,板塊會有兩個下凹轉樞點,第一個轉樞點接近海溝且呈現平緩不明顯的下凹,第二個轉樞點在距海溝幾百公里遠處,曲率明顯、板塊下凹進入深部地幔,如圖1-2(C)所示,阿拉斯加、卡斯卡迪亞(Cascadia)、日本四國與新幾內亞等地區皆屬於此類。

另外在少數區域,隱沒板塊會有三個轉樞點,第一個轉樞點靠近海溝且下凹,第二個轉樞點深度較深呈現上凹,被視為是平坦隱沒的開始端,在這兩個轉樞點中隱沒板塊傾角正常。第三個轉樞點與第二個轉樞點深度相近,水平距離通常超過100公里以上,其曲率下凹的特徵代表著平坦隱沒的結束,平坦隱沒的距離與深度由第二與第三轉樞點所決定,如圖1-2(D)所示,智利、秘魯與墨西哥等地區屬於此類,在過去曾經被 Manea et al., 2017所討論。在本研究中,平坦隱沒被定義為具有三個轉樞點的隱沒板塊。

\section{Review of flat subduction numerical models}

目前造成平坦隱沒發生的機制眾說紛紜。南美洲區域平坦隱沒的發生區域與隱沒的中洋脊有幾何上的相關性,海洋地殼上中洋脊與海洋高原的存在可能會導致總體密度較低、浮力較大,因此過去曾經隱沒的中洋脊被認為是造成平坦隱沒的主要原因。

Hunen et al., 2002最早將模型加入增厚的海洋地殼,以模擬過去智利與秘魯曾經有中洋脊與海洋高原進入隱沒帶中的紀錄。增厚海洋地殼有較低的密度與較大的浮力,其上方岩相需要比原先更大的壓力與更高的溫度才會從玄武岩相變成密度高的榴輝岩,可能使隱沒板塊與周遭地幔沒有顯著密度差而發生平坦隱沒。不過由於該研究模型僅二維,單純加入增厚海洋地殼所呈現的模型雖然能呈現平坦隱沒,但結果是假設第三維上有無限延伸的增厚海洋地殼,現實中增厚的海洋地殼能造成的浮力效應應遠小於二維模型中的結果。Florez-Rodr´ıguez et al., 2019在三維模型中證明了這一點,他們提出若將現在自然界中最大的洋脊隱沒進入地幔,其所提供的浮力也只會造成海洋板塊傾角減少原先的10度。若從自然界中來看,確實有許多區域皆有海脊隱沒的證據,例如勘察加半島(Kamchatka)有皇帝海脊(Emperor Ridge)隱沒、琉球(Ryukyu)有大東海脊(Daito Ridge)隱沒以及馬里亞納(Mariana)與馬庫斯—內克海脊(Marcus-Necker Ridge)隱沒,然而只有秘魯與智利有平坦隱沒的特徵。此外,在墨西哥有平坦隱沒的特徵,然而墨西哥沒有任何海脊或海洋高原的隱沒紀錄,因此增厚的海洋地殼發生平坦隱沒的理論近年來逐漸站不住腳(Schellart, 2020)。

Hunen et al., 2000 使用二維笛卡爾座標數值模型進行祕魯與智利平坦隱沒的模擬。在他們得模型中,唯一能成功演化出平坦隱沒的機制只有海溝後撤迫使大陸岩石圈逆衝到隱沒板塊之上。Liu and Currie, 2016 使用二維模型模擬過去古法拉隆板塊板塊的平坦隱沒機制,他們加入增厚的海洋地殼後並無法觸發平坦隱沒的產生,然而,再加入額外大陸岩石圈的水平速度後,平坦隱沒便能成功再現。Axen et al., 2018使用同樣的數值模型將古代北美西部的克拉通放置於大陸板塊測,成功模擬出增厚海洋地殼加上快速移動大陸岩石圈能發生平坦隱沒,並且能將克拉通從大陸岩石圈底部刮除,證實了平坦隱沒能破壞大陸岩石圈。在該研究中並沒有考慮克拉通對平坦隱沒的影響。

Manea et al., 2012提出了另外的看法。他們利用三維模型模擬過去30Ma以來智利區域的隱沒帶動態行為,使用額外施加的邊界條件強迫智利海溝後撤,發現海溝後撤能夠施加給隱沒板塊的地幔流吸力(suction)不足以讓巨大厚重的海洋板塊變平坦,因此他們在模型上覆板塊加上克拉通,系統性測試從150-300公里厚的大陸岩石圈與海溝距離600-1000公里時隱沒帶下方地幔流產生的動力壓力(dynamic pressure)。他們發現在只有在克拉通與海溝距離約800公里且克拉通厚度大於200公里時平坦隱沒才會生成。當他們把造成海溝後撤的邊界力移除時,不會觸發平坦隱沒的形成,因此他們得出的結論是需要同時有海溝後撤與克拉通的存在才會觸發平坦隱沒。這是首次將克拉通加進數值模型裡的平坦隱沒模型。隨後Liu and Currie, 2016效仿同樣的機制,將過去普遍認為存在於北美板塊西部下方的科羅拉多高原山根放入模型中,模擬古法拉龍板塊平坦隱沒演化。他們認為克拉通與山根的存在只是加快平坦隱沒的形成,但真正觸發平坦隱沒的機制是增厚海洋地殼延緩玄武岩相變成榴輝岩。Hu et al., 2016使用三維模型CitcomS模擬整個南美洲海溝45 Ma以來隱沒帶演化。在加入克拉通的模型中,隱沒板塊傾角有降低的趨勢,不過根據模型結果,真正造成平坦隱沒的形成依然與隱沒海脊相關,只有在海脊進入三維模型後隱沒傾角才出現顯著降低。

因此,目前的平坦隱沒數值模型大多以擬合智利、祕魯與法拉龍板塊為主,觸發平坦隱沒的機制大多與克拉通的存在與否、是否有洋脊隱沒以及上覆板塊的移動速度為主要測試,墨西哥區域尚未有平坦隱沒的數值模型被提出。在墨西哥,隱沒板塊上沒有任何增厚的紀錄,此外該地區北美板塊移動速率遠低於南美洲與過去法拉龍板塊隱沒時期的北美板塊,因此墨西哥區域的平坦隱沒機制尚未有統一定論。本研究期待能利用數值模擬得到墨西哥平坦隱沒從過去50 Ma以來的演化,並提出新的演化機制模型,填補過去尚未成熟的平坦隱沒機制理論。

\section{Movation}

(尚未開始撰寫)

\section{Geophysical observation in Cocos subduction zone}

\section{End}

(尚未開始撰寫)