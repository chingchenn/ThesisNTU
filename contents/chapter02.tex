% !TeX root = ../main.tex

\chapter{研究方法}

\section{運動方程式}

Continuum

In geodynamics modelling, we consider major rock units as continuous geological media. The continuous description is described by field variables such as density, pressure, velocity, strain, etc. Since on long timescales geological unit behave like slowly creep fluids, geodynamics process in the viscous part (mantle) are often referred to as process of geodynamical fluid dynamics. 

The mass conservation equation in Lagrangian form is as follow

\begin{align}
\frac{\partial \sigma_{ij}}{\partial x_j}+\rho g_i = \rho \frac{\partial D_{vi}}{\partial t} 
\end{align}

The proof of eq.(2.1) is as follow

There are totally two unknow: density and velocity.
While in geodynamics modelling, the density variations are small enough to be ignored, which is the result of Boussinesq approximation. The Boussinesq approximation assume that the density is linear proportional to the temperature and the small density variation is then neglected, except the gravity term.

\begin{align}
\rho (T) = \rho_0[1-\alpha (T-T_0)] 
\end{align}

where $\rho_0$ is the reference density at temperature $T_0$ and $\alpha$ is the volumetric thermal expansion coefficient. The boussinesq approximation also represent the incompressible condition, which mean the density of material points does no change with time. 

\begin{align}
\nabla \cdot (\vec v) = 0 
\end{align}

The incompressible continuity equation is broadly used in numerical geodynamic modelling, although on many cases it is rather big simplification. 


Eq. (2.3) is the conservation of mass in our numerical modelling approach.


Momentum Equation

In geodynamics, the time-dependent phenomena involve deformation of continuous media, which is the effect of the balance of internal and external forces that act in these media. So as to relate forces and deformation, an equation of motion may be used --The momentum equation. The momentum equation is a differential equivalent of Newton’s second law to a continuous medium.

$f=ma$

Eulerian Form: $\frac{\partial \sigma_{ij}}{\partial x_j}+\rho g_i = \rho (\frac{\partial v_i}{\partial t}+v_j\frac{\partial v_i}{\partial x_j})$

Lagrangian Form: $\frac{\partial \sigma_{ij}}{\partial x_j}+\rho g_i = \rho \frac{\partial D_{vi}}{\partial t}$

F is the net force acting on the object which can be computed locally. 

We will proof the momentum equation of Lagrangian Form below:

For x-component

\begin{align}
f_x=f_{xA}+f_{xB}+f_{xC}+f_{xD}+f_{xE}+f_{xF}+mg_x 
\end{align}

$f_{xA}- f_{xF}$ are stress-related forces, from the outside of the volume on the resoective boundaries A-F. 
$f_g=m_gx$ is the gravity force.

\begin{align}
f_{xA} = -\sigma_{xxA}\Delta y\Delta z
\end{align}


\section{finite elements method}

finite elements method

\section{FLAC}

what is FLAC

We used the Fast Lagrangian Analysis of Continua (FLAC) technique.

\section{Initial Model}

initial model

\section{Rheological behavior}

rheology

*What is viscous and the viscous rheology of rock*

In the near-surface region, rocks undergo relatively low temperature and, therefore, the Earth's lithosphere easily result in brittle (at low pressure) and plastic (at high pressure) deformation. While in the deep earth, temperature increasing with depth, rocks behave viscous with irreversible deformation. Therefore, if a geodynamics model want to account for a wide range rock properties, the model should consider the elasto-visco-plastic rheology of rocks.





Why care about elastivity and plasticity? (From Introduction of Numerical Modelling)

Rocks behave elastically on a relatively short time scale (<$10^4$ year) and, therefore, modelling of relatively fast processes within the Earth's crust and mantle (e.g. magma intrusion) should take into account the elastic propertied of rocks. 
On the other hane, rocks at cold temperatures can also be subjected to localised brittle (at low pressure) and plastic (at higher pressure) deformation, which leads to shear zones and fracture zones in natural rock complexes.
Therefore, if we want to account for this broad range of geodynamic conditions in our models, we should generally consider the visco-elasto-plastic rheology of rocks and be able to model such a complex rheology with our thermomechanical numerical codes.

\section{Phase change}

phase change

\section{Boundary condition}

kinematic boundary condition

thermal boundary condition
