% !TeX root = ../main.tex

\begin{acknowledgement}

謝謝所有走在科學革命上前端的人,感謝哥白尼與加利略對日心說的貢獻,感謝萊布尼茲串連了微分與積分、感謝牛頓所建立的古典力學、達爾文提出的演化論,感謝普朗克與愛因斯坦解開古典物理的兩朵烏雲,拉開物理學上嶄新的一頁,感謝韋格納提出大陸漂移說,感謝威爾遜與所有板塊構造學說的建構者,這些革命者創造了自然科學上理論與實驗的更多可能。
  
地球科學的書籍放在圖書館左手邊的自然科學區塊,這個小角落只有我一個人,大家多半是來吹冷氣的,四周充滿著敲打鍵盤的聲音。  
我自己的書通常都有許多摺痕、或者四個角已經都不是直角,但是圖書館的書太乾淨了,讓人產生了距離感。我敢說這裡一定有些書一輩子都沒有人來翻過。
地球科學研究史說不上太長,書上

花了一點時間從地槽學說到板塊構造學說看了一次,看到百年來科學家一步一步,從藉著氣候提出大陸漂移構想,到由於科技進步得的古磁力證據,最終提出板塊構造學說。手上的書每一頁所呈現的知識都很鋒利,就如同沒有被翻開過的新書紙一樣鋒利,書從外觀到觀念都很沉重,以至於我換了一個姿勢拿著它。看著書上的Morgan、Parker等人,雖然我完全不認識他們,但是藉著拿在手上的書我感受到他們的體溫。  
  走下圖書館的樓梯,回頭看一下自然科學區塊仍然沒有半個人影,看著窗外,陽光照在路上的行道樹,喧囂的車群隨著紅綠燈而停停走走,蟬鳴聲此起彼落。地球正在旋轉,中洋脊正在冒出岩漿,新的板塊正在形成,舊的板塊正在隱沒,斷層時不時發生小錯動,大地不斷邁向均衡,上帝創造這個地球,透過動態的機制達成穩定。人類僅透過秩序達成交通穩定。  
  沒有人知道那個小角落藏著這些書。沒有人知道那個小角落藏著多麼美的事物。

\end{acknowledgement}