% !TeX root = ../main.tex

\begin{abstract}

平坦隱沒(flat slab subduction)是一種特殊的隱沒系統,其隱沒板塊在一固定深度內維持近乎水平狀態持續上百公里,與一般隱沒帶有所區別。
現今自然界中的平坦隱沒發生在科克斯隱沒帶的墨西哥區域以及納茲卡隱沒帶的秘魯與智利區域,其確切形成原因尚未釐清。
儘管兩個隱沒系統板塊幾何相似,然而兩處的平坦深度有顯著的不同,在科克斯隱沒帶,平坦深度與莫荷面相當,僅45公里,而納茲卡隱沒帶的平坦深度可達100公里。
墨西哥平坦隱沒上方有多個活火山存在,且隱沒系統脫水作用活躍,為弱耦合隱沒帶; 然而秘魯與智利平坦隱沒區域自全新世以來便沒有火山活動,該區域多次的大規模地震事件表明其為強耦合隱沒帶。
隱沒板塊的物理特性直接影響隱沒系統的重力力矩,在正常的隱沒帶中,隱沒板塊的密度在玄武岩相變成榴輝岩後急遽增加,地殼與地函顯著的密度差產生重力,形成隱沒系統穩定的驅動力。
過去的平坦隱沒數值模型研究多半使用降低隱沒系統驅動力的方式達成平坦隱沒,因此鮮少將榴輝岩相變加入數值模型中。
隱沒板塊上方地函楔(mantle wedge)的動水壓力(hydrodynamic pressure)降低會大幅增加隱沒系統的吸力力矩(suction torque)量值,倘若吸力力矩能超越重力力矩對隱沒系統施加的影響,可能是形成平坦隱沒的重要原因之一,然而由於影響動水壓力的因素難以掌控,平坦隱沒的發生原因與時機還是一個開放性問題。

本研究利用地球動力學二維數值模型FLAC (Fast Lagrangian Analysis of Continua)分別模擬科克斯隱沒帶與納茲卡隱沒帶的平坦隱沒。
利用參數化模擬部分熔融與岩漿庫的效應以實現隱沒系統中的岩漿作用,並用以討論墨西哥與智利兩種截然不同的岩漿作用特徵。
研究結果顯示智利平坦隱沒模型需要一個蛇紋岩化橄欖岩較少的地函楔環境。
狹窄的地函楔蛇紋岩化橄欖岩在聚合板塊交界處形成細窄的低黏滯度通道,導致隱沒板塊上下地函的壓力差增加,進而增加隱沒系統中的吸力力矩,形成平坦隱沒。

智利模型的平坦隱沒長度略少於實際長度,然而隱沒板塊平坦段深度與觀測結果相符合。
該模型在運行的時間段內皆有持續的部分熔融事件發生,然而由於模型的上覆板塊溫度較低,因此岩漿庫存在時間不長,導致該模型並沒有火山島弧的出現。
岩漿來源多為橄欖岩部分熔融,少數的隱沒沉積物有熔融跡象。
智利區域的岩漿組成份分析認為平坦隱沒上方火成岩具有埃達克岩(adakites)特徵,然而其形成機制確切原因還有待商榷。
墨西哥平坦隱沒模型則需要大量沉積物與蛇紋岩化橄欖岩的存在,且具備溫暖的上覆板塊與較快速的聚合速率。
該模型的平坦深度同樣吻合目前地震學觀測結果,平坦隱沒長度略少於實際長度。
模型中的部分熔融源在平坦隱沒生成之前為普通的橄欖岩,然而在平坦隱沒生成之後,決大部分熔融源皆為隱沒板塊上的沉積物與玄武岩物質,該結果與墨西哥平坦隱沒上方的埃達克岩紀錄相吻合。


\end{abstract}

\begin{abstract*}

Flat slab subduction, where the subducting slab moves sub-horizontally for hundreds of kilometres before diving into deeper mantle, is one of the most famous unusual subduction processes.
Flat slab subduction is recognised only in the Cocos and Nazca Plates in the present-day.
The reasons that the Cocos (Mexico) and Nazca (Chile and Peru) subduction zones develop from steep to flat subduction are not well understood.
While the geometry of the subducting plates is similar in these two regions, the flat slab depth is 45 km near the Moho in Cocos subduction zone and 100 km under lithosphere in Nazca subduction zone, respectively.
By the observation, Mexico flat slab subduction has a weak interface between the subducting plate and the upper plate, which leads to weak coupling.
Chile flat slab subduction generates a compression state on the upper plate, showing a strong coupling along the subducting plate.
The magmatism process is also different in Mexico and Chile.
While the volcano is still active above Cocos subdiction zone, the last active volcano above Nazca subduction occurred in the Miocene.
Adakites rocks are found in the far trench inland in both of these subduction zone.
There were many disputes over the formation mechanism between these two regions.

The purpose of this study is to explore flat slab subduction processes by using 2D thermo-mechanical models with Fast Lagrangian Analysis of Continua (FLAC) technology. 
We considered rock rheology of both elasto-plastic and visco-elastic deformations. 
To achieve the dehydration process during subduction, we parameterized the serpentinized peridotite by transforming the peridotite into a fixed thickness and realised the magma formation by tracking the P-T (pressure-temperature) path of markers.

Our results indicate the Chile flat subduction model requires a relatively small amount of serpentinized peridotites, which generates a thin low viscosity channel and a large pressure contrast between sub-slab mantle and mantle wedge, resulting in high suction force in the subduction system. 
The strong compression stress occurs in the top of the upper plate and subduction interface shows a strong coupling in the subduction system. 
Through the partial melting occurs continuously, the cold continental lithosphere leads to the fast decay of magma chamber. 
Therefore, the Chile flat slab model does not show any volcanic arc on the surface. 
By contrast, the Mexico flat slab subduction requires a warm overriding plate with high convergence rate and a large amount of serpentinite on the top of the slab. 
The upper plate does not undergo any compression or extension process in the model. 
The melting rocks in Mexico flat slab model change with time. 
The peridotite material is the predominant source of the magma chamber before the flat slab subduction develops.
Sediment and basalt start to melt once the flat slab occurs, implying the existence of adakites rocks. 
Our study suggests that the formation mechanism of flat slab subduction is different between Cocos and Nazca subduction zones.
    
\end{abstract*}